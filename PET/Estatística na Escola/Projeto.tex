\documentclass[
	12pt,				% tamanho da fonte
	openright,			% capítulos começam em pág ímpar (insere página vazia caso preciso)
	oneside,			% para impressão apenas em um lado do papel
	a4paper,			% tamanho do papel.
	brazil				% o último idioma é o principal do documento
	]{abntex2}

% ---
% Pacotes básicos
% ---

\usepackage{lmodern}			% Usa a fonte Latin Modern			
\usepackage[T1]{fontenc}		% Selecao de codigos de fonte.
\usepackage[utf8]{inputenc}		% Codificacao do documento (conversão automática dos acentos)
\usepackage{lastpage}			% Usado pela Ficha catalográfica
\usepackage{indentfirst}		% Indenta o primeiro parágrafo de cada seção.
\setlength{\parindent}{1.5cm}   % Espaçamento de 1,5cm do parágrafo
\usepackage{color}				% Controle das cores
\usepackage{graphicx}			% Inclusão de gráficos
\usepackage{microtype} 			% para melhorias de justificação
\usepackage{lipsum}				% para geração de dummy text
\usepackage[alf]{abntex2cite}	% Citações padrão ABNT
\usepackage[table,xcdraw]{xcolor}% Cédula colorida em tabelas
\usepackage{pdflscape}          % Rotaciona página
\usepackage{Capa}               % Capa e folha de rosto com as modificações da Suely

% ---
% Dados dos documento
% ---

\titulo{Projeto de extensão}
\autor{Nivea Zamaro \\ Willian Meira}
\data{2018}
\instituicao{Universidade Federal do Paraná
             \par
             Setor de Ciências Exatas
             \par
             Departamento de Estatística
%             \par
%             Curso de Estatística
            }
\local{Curitiba}
\orientador[Orientadora:]{Profa. Dra. Suely Ruiz Giolo}
\preambulo{Projeto de Pesquisa apresentado à disciplina Laboratório A do Curso de Graduação em Estatística da
           Universidade Federal do Paraná, como requisito para elaboração do Trabalho de Conclusão de Curso.}

% ---
% Configurações básicas
% ---

% informações do PDF

\makeatletter
\hypersetup{
     	%pagebackref=true,
		pdftitle={\@title},
		pdfauthor={\@author},
    	pdfsubject={\imprimirpreambulo},
		colorlinks=true,       		% false: boxed links; true: colored links
    	linkcolor=black,          	% color of internal links
    	citecolor=black,            % color of links to bibliography
    	filecolor=magenta,      	% color of file links
		urlcolor=black,
		bookmarksdepth=4
}
\makeatother

\setlength\afterchapskip{\lineskip}

% ----
% Início do documento
% ----

\begin{document}
\frenchspacing     % Retira espaço extra obsoleto entre as frases.

% ----------------------------------------------------------
% ELEMENTOS PRÉ-TEXTUAIS
% ----------------------------------------------------------

% ---
% Capa
% ---
\imprimircapa
% ---

% ---
% Folha de rosto
% ---
\imprimirfolhaderosto
% ---

% ---
% inserir o sumario
% ---
\tableofcontents*
\cleardoublepage
% ---

\makepagestyle{abntheadings}
\makeevenhead{abntheadings}{\ABNTEXfontereduzida\thepage}{}{}
\makeoddhead{abntheadings}{}{}{\ABNTEXfontereduzida\thepage}
\makeheadrule{abntheadings}{\textwidth}{0in}

% ----------------------------------------------------------
% ELEMENTOS TEXTUAIS
% ----------------------------------------------------------
\textual

\chapter{Introdução}
\bigskip

Digite a introdução do projeto.

\bigskip 
% comandos para citar no texto uma referencia do arquivo Referencias.bib
  
O estimador proposto por Kaplan e Meier \cite{kaplan} é .... 

\citeonline{abadi}, por exemplo, realizaram um estudo....  

\chapter{Objetivos}

{\section{Objetivos Gerais}}

Analisar os dados do  .......

\section{Objetivos Específicos}

\begin{alineas}
\item Identificar .... ;
\item Estudar ... ;
\item Discutir ....
\end{alineas}

\chapter{Material e Métodos}
\bigskip

Digitar paragrafo introdutório ....

\section{Material}

Descrever os dados e \textit{softwares} a serem utilizados para a análise dos dados

\section{Métodos}

Descrever brevemente os métodos os quais se pretende utilizar

\begin{landscape}    % página na orientação paisagem
\chapter{Cronograma de Atividades}

\begin{table}[h]
\begin{tabular}{lllllll}
\hline
 & \textbf{ATIVIDADES} & \textbf{02/2016} & \textbf{03/2016} & \textbf{04/2016} & \textbf{05/2016} & \textbf{06/2016} \\ \hline
\textbf{1} & \textbf{Projeto de Pesquisa} &  &  &  &  &  \\
 & Entrega da versão final do Projeto de Pesquisa ao orientador & \cellcolor[HTML]{C0C0C0}{\color[HTML]{C0C0C0} } &  &  &  &  \\
\textbf{2} & \textbf{Elaboração do Trabalho de Conclusão de Curso} &  &  &  &  &  \\
 & Revisão de literatura sobre o tema & \cellcolor[HTML]{C0C0C0} & \cellcolor[HTML]{C0C0C0} &  &  &  \\
 & Análise dos dados e discussão dos resultados obtidos &  & \cellcolor[HTML]{C0C0C0} & \cellcolor[HTML]{C0C0C0} & \cellcolor[HTML]{C0C0C0} &  \\
 & Redação do trabalho de conclusão de curso &  &  & \cellcolor[HTML]{C0C0C0} & \cellcolor[HTML]{C0C0C0} &  \\
 & Leitura do trabalho pelo orientador e correções &  &  &  & \cellcolor[HTML]{C0C0C0} & \cellcolor[HTML]{C0C0C0} \\
 & Entrega do trabalho redigido aos membros da banca &  &  &  &  & \cellcolor[HTML]{C0C0C0} \\
\textbf{3} & \textbf{Defesa do Trabalho de Conclusão de Curso} &  &  &  &  &  \\
 & Preparação e apresentação do trabalho de conclusão de curso &  &  &  &  & \cellcolor[HTML]{C0C0C0} \\
\textbf{4} & \textbf{Elaboração da Versão Final do Trabalho de Conclusão de Curso} &  &  &  &  &  \\
 & Elaboração da versão final do TCC &  &  &  &  & \cellcolor[HTML]{C0C0C0} \\
 & Entrega da versão final do trabalho ao orientador &  &  &  &  & \cellcolor[HTML]{C0C0C0} \\ \hline
\end{tabular}
\end{table}
\end{landscape}
% ----------------------------------------------------------
% ELEMENTOS PÓS-TEXTUAIS
% ----------------------------------------------------------
\postextual
% ----------------------------------------------------------

% ----------------------------------------------------------
% Referências bibliográficas
% ----------------------------------------------------------

\setlength{\afterchapskip}{\baselineskip}
\bibliography{Referencias}

\end{document} 